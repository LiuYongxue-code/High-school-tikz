%% symbol library for TikZ track schematics
%
% Copyright (c) 2018 - 2020, Martin Scheidt (ISC license)
%
% Permission to use, copy, modify, and/or distribute this file for any purpose with or without fee is hereby granted, provided that the above copyright notice and this permission notice appear in all copies.
%
\ProvidesFileRCS{tikzlibrarytrackschematic.trafficcontrol.code.tex}%
%
%%%%%%%%%%%%%%%
% Requirements
%%%%%%%%%%%%%%%
\RequirePackage{tikz,etoolbox,lmodern}%
\usetikzlibrary{calc}%
%
% https://tex.stackexchange.com/questions/56353/extract-x-y-coordinate-of-an-arbitrary-point-on-curve-in-tikz
\providecommand{\gettikzxy}[3]{%
  \tikz@scan@one@point\pgfutil@firstofone#1\relax%
  \edef#2{\the\pgf@x}%
  \edef#3{\the\pgf@y}%
}%
%
%%%%%%%%%%%%%%%
% tikz keys for multiple use
%%%%%%%%%%%%%%%
\pgfkeys{%
  /tikz/trackschematic/.is family,%
  /tikz/trackschematic/.cd,%
  %% color \foreground
  foreground/.store in=\foreground,%
  foreground=black,% DEFAULT
  /tikz/foreground/.forward to=/tikz/trackschematic/foreground,%
  %% color \background
  background/.store in=\background,%
  background=white,% DEFAULT
  /tikz/background/.forward to=/tikz/trackschematic/background,%
  %% face
  face/.value required,% forward OR backward
  face/.store in=\face,% forward OR backward
  /tikz/face/.forward to=/tikz/trackschematic/face,%
  /tikz/forward/.code={\pgfkeys{/tikz/trackschematic/face=forward}},%
  /tikz/backward/.code={\pgfkeys{/tikz/trackschematic/face=backward}},%
  /tikz/bidirectional/.code={\pgfkeys{/tikz/trackschematic/face=bidirectional}},%
  %% traffic practice
  traffic practice/.value required,% left OR right
  traffic practice/.store in=\trafficpractice,%
  traffic practice=right,% DEFAULT
  /tikz/traffic practice/.forward to=/tikz/trackschematic/traffic practice,%
  /tikz/position/.forward to=/tikz/trackschematic/traffic practice,%
  %% label
  shift label/.store in=\labelcoord,% (coord)
  shift label=(none),% DEFAULT
  /tikz/shift label/.forward to=/tikz/trackschematic/shift label,%
}%
%
\pgfkeys{%
  /tikz/trackschematic/trafficcontrol/.is family,%
  /tikz/trackschematic/trafficcontrol/.cd,%
  %% type
  type/.value required,% balise OR loop
  type/.store in=\type,% balise OR loop
  /tikz/type/.forward to=/tikz/trackschematic/trafficcontrol/type,%
  %% block signal type
  block type/.value forbidden,%
  block type/.code={\settoggle{is_block_type}{true}},%
  /tikz/block/.forward to=/tikz/trackschematic/trafficcontrol/block type,%
  %% route signal type
  route type/.value forbidden,%
  route type/.code={\settoggle{is_route_type}{true}},%
  /tikz/route/.forward to=/tikz/trackschematic/trafficcontrol/route type,%
}%
% options
\newtoggle{is_block_type}\settoggle{is_block_type}{false}%
\newtoggle{is_route_type}\settoggle{is_route_type}{false}%
%%%%%%%%%%%%%%%
% symbol signal
%%%%%%%%%%%%%%%
% command
\newcommand\signal{}% just for safety
\def\signal[#1]#2(#3)#4(#5){% \signal[options] at (coord) label (name);
  \pic[#1] at (#3) {signal={#2/#4/#5}}% symbol
}%
\newcommand\distantsignal{}% just for safety
\def\distantsignal[#1]#2(#3)#4(#5){% \distantsignal[options] at (coord) label (name);
  \pic[distant,#1] at (#3) {signal={#2/#4/#5}}% symbol
}%
\newcommand\speedsign{}% just for safety
\def\speedsign[#1]#2(#3)#4(#5){% \speedsign[options] at (coord) label (name);
  \pic[speed type,#1] at (#3) {signal={#2/#4/#5}}% symbol
}%
\newcommand\speedsignal{}% just for safety
\def\speedsignal[#1]#2(#3)#4(#5){% \speedsignal[options] at (coord) label (name);
  \pic[speed type,#1] at (#3) {signal={#2/#4/#5}}% symbol
}%
\newcommand\speeddistantsignal{}% just for safety
\def\speeddistantsignal[#1]#2(#3)#4(#5){% \speedsignal[options] at (coord) label (name);
  \pic[speed type,distant,#1] at (#3) {signal={#2/#4/#5}}% symbol
}%
\newcommand\blocksignal{}% just for safety
\def\blocksignal[#1]#2(#3)#4(#5){% \blocksignal[options] at (coord) label (name);
  \pic[block,#1] at (#3) {signal={#2/#4/#5}}% symbol
}%
\newcommand\routesignal{}% just for safety
\def\routesignal[#1]#2(#3)#4(#5){% \routesignal[options] at (coord) label (name);
  \pic[route,#1] at (#3) {signal={#2/#4/#5}}% symbol
}%
\newcommand\shuntsignal{}% just for safety
\def\shuntsignal[#1]#2(#3)#4(#5){% \shuntsignal[options] at (coord) label (name);
  \pic[shunting,#1] at (#3) {signal={#2/#4/#5}}% symbol
}%
\newcommand\shuntlimit{}% just for safety
\def\shuntlimit[#1]#2(#3)#4(#5){% \shuntlimit[options] at (coord) label (name);
  \pic[shunt limit,#1] at (#3) {signal={#2/#4/#5}}% symbol
}%
\newcommand\berthsignal{}% just for safety
\def\berthsignal[#1]#2(#3)#4(#5){% \berthsignal[options] at (coord) label (name);
  \pic[berth,#1] at (#3) {signal={#2/#4/#5}}% symbol
}%
\newcommand\berthsign{}% just for safety
\def\berthsign[#1]#2(#3)#4(#5){% \berthsign[options] at (coord) label (name);
  \pic[berth,#1] at (#3) {signal={#2/#4/#5}}% symbol
}%
% tikz keys
\pgfkeys{%
  /tikz/trackschematic/trafficcontrol/signal/.is family,%
  /tikz/trackschematic/trafficcontrol/signal/.cd,%
  %% distant signal type
  distant type/.value forbidden,%
  distant type/.code={\settoggle{is_distant_type}{true}},%
  /tikz/distant/.forward to=/tikz/trackschematic/trafficcontrol/signal/distant type,%
  %% block signal type
  speed type/.value forbidden,%
  speed type/.code={\settoggle{is_speed_type}{true}},%
  /tikz/speed type/.forward to=/tikz/trackschematic/trafficcontrol/signal/speed type,%
  %% shunting signal type
  shunting type/.value forbidden,%
  shunting type/.code={\settoggle{is_shunting_type}{true}},%
  /tikz/shunting/.forward to=/tikz/trackschematic/trafficcontrol/signal/shunting type,%
  %% shunting signal type
  shunt limit/.value forbidden,%
  shunt limit/.code={\settoggle{is_shunt_limit}{true}},%
  /tikz/shunt limit/.forward to=/tikz/trackschematic/trafficcontrol/signal/shunt limit,%
  %% berth signal type
  berth type/.value forbidden,%
  berth type/.code={\settoggle{is_berth_type}{true}},%
  /tikz/berth/.forward to=/tikz/trackschematic/trafficcontrol/signal/berth type,%
  %% speed value
  speed/.store in=\speed,% number
  speed=,% number
  /tikz/speed/.forward to=/tikz/trackschematic/trafficcontrol/signal/speed,%
  %% speed value
  distant speed/.store in=\distantspeed,% number
  distant speed=,% number
  /tikz/distant speed/.forward to=/tikz/trackschematic/trafficcontrol/signal/distant speed,%
  %% locked signal
  locked/.value forbidden,%
  locked/.code={\settoggle{is_locked}{true}},%
  /tikz/locked/.forward to=/tikz/trackschematic/trafficcontrol/signal/locked,%
}%
% options
\newtoggle{is_distant_type}\settoggle{is_distant_type}{false}%
\newtoggle{is_speed_type}\settoggle{is_speed_type}{false}%
\newtoggle{is_shunting_type}\settoggle{is_shunting_type}{false}%
\newtoggle{is_shunt_limit}\settoggle{is_shunt_limit}{false}%
\newtoggle{is_berth_type}\settoggle{is_berth_type}{false}%
\newtoggle{is_locked}\settoggle{is_locked}{false}%
% symbol definition
\tikzset{% generic symbol
  pics/signal/.default=,%
  pics/signal/.style args={#1/#2/#3}{code={%
    %% settings
    \def\coordcommand{#1}% beware of leading and tailing spaces!
    \def\labelcommand{#2}% beware of leading and tailing spaces!
    \def\labelcontent{#3}%
    %
    %% traffic practice setup
    \ifdefstring{\trafficpractice}{left}{% branch
      \pgfmathsetmacro{\trafficfactor}{-1}%
    }{%
      \ifdefstring{\trafficpractice}{right}{% branch
        \pgfmathsetmacro{\trafficfactor}{1}%
      }{% error message
        \pgfkeys{/errors/unknown choice value={/tikz/trackschematic/trafficcontrol/traffic practice}{“left“ OR “right“ as key required}}%
      }%
    }% end \ifdefstring{\trafficpractice}
    %% face setup
    \ifdefstring{\face}{forward}{% face
      \pgfmathsetmacro{\facefactor}{1}%
      \def\align{left}%
      \def\rotate{-90}%
    }{%
      \ifdefstring{\face}{backward}{% face
        \pgfmathsetmacro{\facefactor}{-1}%
        \def\align{right}%
        \def\rotate{90}%
      }{% error message
        \pgfkeys{/errors/unknown choice value={/tikz/trackschematic/face}{“forward“ OR “backward“ as key required}}%
      }%
    }% end \ifdefstring{\face}
    \tikzset{every path/.style={draw=\foreground,line width=1pt}};%
    \tikzset{every node/.style={text=\foreground,inner sep=1pt}};%
    %% signal pole
    \path (0,0) -- ++($\trafficfactor*\facefactor*(0,-0.4)$) -- ++($\facefactor*(0.7,0)$);% signal pole
    %
    %% label
    \ifdefstring{\labelcontent}{}{}{% label NOT empty
      \coordinate (ts-s-l) at ($\trafficfactor*\facefactor*(0,-0.4)$);%
      \ifdefstring{\labelcoord}{(none)}{}{% initialize if NOT default
        \gettikzxy{\labelcoord}{\labelcoordX}{\labelcoordY}%
        \coordinate (ts-s-l) at ($(ts-s-l)+(\labelcoordX,\labelcoordY)$);%
      }%
      \node[\align] at (ts-s-l) {\footnotesize \labelcontent};%
    }%
    \tikzset{every path/.style={draw=\foreground,line width=1pt,fill=\background},rounded corners=0.1pt};%
    %% signal marker
    \iftoggle{is_distant_type}{% marker for distant signal
      \path ($\trafficfactor*\facefactor*(0,-0.4) + \facefactor*(0.35,0)$) --%
          ++($\trafficfactor*\facefactor*(0,-0.2) + \facefactor*(0.35,0)$) --%
          ++($\trafficfactor*\facefactor*(0, 0.4)$) -- cycle;% signal marker
    %
    }{}%
    \iftoggle{is_speed_type}{% marker for speed signal
      \path ($\trafficfactor*\facefactor*(0,-0.2) + \facefactor*(0.35,0)$) --%
          ++($\trafficfactor*\facefactor*(0,-0.4)$) --%
          ++($\trafficfactor*\facefactor*(0, 0.2) + \facefactor*(0.35,0)$) -- cycle;% signal marker
      %
    }{}%
    \iftoggle{is_block_type}{% marker for block signal
      \path ($\trafficfactor*\facefactor*(0,-0.6) + \facefactor*(0.7,0)$) rectangle%
          ++($\trafficfactor*\facefactor*(0, 0.4) + \facefactor*(0.4,0)$);% signal marker
      %
    }{}%
    \iftoggle{is_route_type}{% marker for route signal
      \path ($\trafficfactor*\facefactor*(0,-0.4) + \facefactor*(0.9,0)$) circle (0.2);% signal marker
      %
    }{}%
    \iftoggle{is_shunting_type}{% marker for shunting signal
      \path ($\trafficfactor*\facefactor*(0,-0.3) + \facefactor*(0.6,0)$) circle (0.1);% signal marker
      %
    }{}%
    \iftoggle{is_shunt_limit}{% marker for shunting signal
      \tikzset{semicircle/.pic={\path (0,0) arc (180:0:0.1) -- cycle;};}%
      \pgfmathsetmacro{\trafficfactorTEST}{-1}%
      \ifdefequal{\trafficfactor}{\trafficfactorTEST}{%
        \pgfmathsetmacro{\trafficfactorX}{-2}%
      }{%
        \pgfmathsetmacro{\trafficfactorX}{1}%
      }%
      \pic[rotate=\rotate] at ($\trafficfactorX*\facefactor*(0,-0.2) + \facefactor*(0.6,0)$) {semicircle}; % signal marker
      %
    }{}%
    \iftoggle{is_berth_type}{% marker for berth signal
      \path ($\trafficfactor*\facefactor*(0,-0.575) + \facefactor*(0.3,0)$) rectangle%
          ++($\trafficfactor*\facefactor*(0, 0.35 ) + \facefactor*(0.5,0)$);% % signal marker
      \path[line width=0.75pt] ($\trafficfactor*\facefactor*(0,-0.3) + \facefactor*(0.375,0)$) -- ++($\facefactor*(0.35,0)$);%
      \path[line width=0.75pt] ($\trafficfactor*\facefactor*(0,-0.5) + \facefactor*(0.55 ,0)$) -- ++($\trafficfactor*\facefactor*(0,0.2)$);%
      \path[line width=0.75pt] ($\trafficfactor*\facefactor*(0,-0.5) + \facefactor*(0.375,0)$) -- ++($\facefactor*(0.35,0)$);%
      %
    }{}%
    %% speed indicator
    \ifdefstring{\speed}{}{}{% speed NOT empty
      \tikzset{every node/.style={font=\sffamily,text=\foreground}};%
      \iftoggle{is_speed_type}{% marker for speed signal
        \node[rotate=\rotate] at ($\trafficfactor*\facefactor*(0,-0.4) + \facefactor*(0.85,0)$) {\speed};%
      }{%
        \iftoggle{is_shunting_type}{}{% is NOT shunting tyoe
          \node[rotate=\rotate] at ($\trafficfactor*\facefactor*(0,-0.4) + \facefactor*(1.3,0)$) {\speed};%
        }%
      }%
    }%
    \ifdefstring{\distantspeed}{}{}{% distant speed NOT empty
      \tikzset{every node/.style={font=\sffamily,text=\foreground,fill=\background,inner sep=0.5pt}};%
      \node[rotate=\rotate] at ($\trafficfactor*\facefactor*(0,-0.4) + \facefactor*(0.2,0)$) {\distantspeed};%
    }%
    %% locked
    \iftoggle{is_locked}{% marker for route signal
      \iftoggle{is_block_type}{% marker for block signal
        \path ($\trafficfactor*\facefactor*(0,-0.2) + \facefactor*(0.9,0)$) --%
            ++($\trafficfactor*\facefactor*(0,-0.4)$);% signal aspect
      }{}%
      \iftoggle{is_route_type}{% marker for route signal
        \path ($\trafficfactor*\facefactor*(0,-0.2) + \facefactor*(0.9,0)$) --%
            ++($\trafficfactor*\facefactor*(0,-0.4)$);% signal aspect
      }{}%
      \iftoggle{is_shunting_type}{%
        \path ($\trafficfactor*\facefactor*(0,-0.2) + \facefactor*(0.6,0)$) --%
            ++($\trafficfactor*\facefactor*(0,-0.2)$);% signal aspect
      }{}%
    }{}%
  }},% end pics/signal/.style args={#1/#2/#3}
}%
%
%%%%%%%%%%%%%%%
% symbol clearing point
%%%%%%%%%%%%%%%
% command
\newcommand\clearingpoint{}% just for safety
\def\clearingpoint[#1]#2(#3)#4(#5){% \clearingpoint[options] at (coord) label (name);
  \pic[standard,#1] at (#3) {clearing_point={#2/#4/#5}}% symbol
}%
\newcommand\blockclearing{}% just for safety
\def\blockclearing[#1]#2(#3)#4(#5){% \blockclearing[options] at (coord) label (name);
  \pic[block,#1] at (#3) {clearing_point={#2/#4/#5}}% symbol
}%
\newcommand\routeclearing{}% just for safety
\def\routeclearing[#1]#2(#3)#4(#5){% \routeclearing[options] at (coord) label (name);
  \pic[route,#1] at (#3) {clearing_point={#2/#4/#5}}% symbol
}%
\pgfkeys{%
  /tikz/trackschematic/trafficcontrol/clearing point/.is family,%
  /tikz/trackschematic/trafficcontrol/clearing point/.cd,%
  %% standard type
  standard type/.value forbidden,%
  standard type/.code={\settoggle{is_standard_type}{true}},%
  /tikz/standard/.forward to=/tikz/trackschematic/trafficcontrol/clearing point/standard type,%
}%
% options
\newtoggle{is_standard_type}\settoggle{is_standard_type}{false}%
% symbol definition
\tikzset{%
  pics/clearing_point/.default=,%
  pics/clearing_point/.style args={#1/#2/#3}{code={%
    %% settings
    \def\coordcommand{#1}% beware of leading and tailing spaces!
    \def\labelcommand{#2}% beware of leading and tailing spaces!
    \def\labelcontent{#3}%
    %
    %% traffic practice setup
    \ifdefstring{\trafficpractice}{left}{% branch
      \pgfmathsetmacro{\trafficfactor}{-1}%
    }{%
      \ifdefstring{\trafficpractice}{right}{% branch
        \pgfmathsetmacro{\trafficfactor}{1}%
      }{% error message
        \pgfkeys{/errors/unknown choice value={/tikz/trackschematic/trafficcontrol/traffic practice}{“left“ OR “right“ as key required}}%
      }%
    }% end \ifdefstring{\trafficpractice}
    %% face setup
    \ifdefstring{\face}{backward}{% face
      \pgfmathsetmacro{\facefactor}{-1}%
    }{% default case
      \pgfmathsetmacro{\facefactor}{1}%
    }% end \ifdefstring{\face}
    \tikzset{every path/.style={draw=\foreground,line width=1pt}};%
    %% marker
    \path ($\trafficfactor*\facefactor*(0,-0.1)$) -- ++($\trafficfactor*\facefactor*(0,0.2)$);% marker
    %% sign
    \iftoggle{is_standard_type}{% marker for block signal
      \path ($\trafficfactor*\facefactor*(0,-0.1) + \facefactor*(0.1,0)$) -- ++($\facefactor*(-0.2,0)$);% sign
    }{}%
    \iftoggle{is_block_type}{% marker for block signal
      \path ($\trafficfactor*\facefactor*(0,-0.1)$) --%
          ++($\trafficfactor*\facefactor*(0,-0.1) + \facefactor*(-0.1,0)$) --%
          ++($\trafficfactor*\facefactor*(0,-0.1) + \facefactor*( 0.1,0)$) --%
          ++($\trafficfactor*\facefactor*(0, 0.1) + \facefactor*( 0.1,0)$) -- cycle;% sign
    }{}%
    \iftoggle{is_route_type}{% marker for route signal
      \path ($\trafficfactor*\facefactor*(0,-0.2)$) circle (0.1);% sign
    }{}%
    %% label
    \ifdefstring{\labelcontent}{}{}{% label NOT empty
      \tikzset{every node/.style={font=\sffamily,text=\foreground}};%
      \coordinate (ts-cp-l) at ($\trafficfactor*\facefactor*(0,0.25)$);%
      \ifdefstring{\labelcoord}{(none)}{}{% initialize if NOT default
        \gettikzxy{\labelcoord}{\labelcoordX}{\labelcoordY}%
        \coordinate (ts-cp-l) at ($(ts-cp-l)+(\labelcoordX,\labelcoordY)$);%
      }%
      \node at (ts-cp-l) {\footnotesize \labelcontent};%
    }%
  }},% end pics/clearing_point/.style args={#1/#2/#3}
}%
%
%%%%%%%%%%%%%%%
% symbol transmitter
%%%%%%%%%%%%%%%
% command
\newcommand\transmitter{}% just for safety
\def\transmitter[#1]#2(#3)#4(#5){% \transmitter[options] at (coord) label (name);
  \pic[#1] at (#3) {transmitter={#2/#4/#5}}% symbol
}%
\newcommand\balise{}% just for safety
\def\balise[#1]#2(#3)#4(#5){% \balise[options] at (coord) label (name);
  \pic[type=balise,#1] at (#3) {transmitter={#2/#4/#5}}% symbol
}%
\newcommand\trackloop{}% just for safety
\def\trackloop[#1]#2(#3)#4(#5){% \trackloop[options] at (coord) label (name);
  \pic[type=loop,#1] at (#3) {transmitter={#2/#4/#5}}% symbol
}%
% symbol definition
\tikzset{%
  pics/transmitter/.default=,%
  pics/transmitter/.style args={#1/#2/#3}{code={%
    %% settings
    \def\coordcommand{#1}% beware of leading and tailing spaces!
    \def\labelcommand{#2}% beware of leading and tailing spaces!
    \def\labelcontent{#3}%
    %% traffic practice setup
    \ifdefstring{\trafficpractice}{left}{% branch
      \pgfmathsetmacro{\trafficfactor}{-1}%
    }{%
      \ifdefstring{\trafficpractice}{right}{% branch
        \pgfmathsetmacro{\trafficfactor}{1}%
      }{% error message
        \pgfkeys{/errors/unknown choice value={/tikz/trackschematic/trafficcontrol/traffic practice}{“left“ OR “right“ as key required}}%
      }%
    }% end \ifdefstring{\trafficpractice}
    %% marker
    \tikzset{every path/.style={draw=\foreground}};%
    \ifdefstring{\type}{balise}{% type balise
      \path[line width=1pt,fill=\background] ($(-0.25,0)$) rectangle%
                                             ($\trafficfactor*(0,-0.25) + (0.25,0)$);% balise marker
      \ifdefstring{\face}{forward}{% face
        \path ($\trafficfactor*(0,-0.05) + (0.1,0)$) -- ($\trafficfactor*(0,-0.125) + (0.2,0)$) --%
              ($\trafficfactor*(0,-0.2) + (0.1,0)$) -- cycle;% arrow forward
      }{%
        \ifdefstring{\face}{backward}{% face
          \path ($\trafficfactor*(0,-0.05) + (-0.1,0)$) -- ($\trafficfactor*(0,-0.125) + (-0.2,0)$) --%
                ($\trafficfactor*(0,-0.2) + (-0.1,0)$) -- cycle;% arrow backward
        }{
          \ifdefstring{\face}{bidirectional}{% face
            \path ($\trafficfactor*(0,-0.05) + (0.1,0)$) -- ($\trafficfactor*(0,-0.125) + (0.2,0)$) --%
                  ($\trafficfactor*(0,-0.2) + (0.1,0)$) -- cycle;% arrow forward
            \path ($\trafficfactor*(0,-0.05) + (-0.1,0)$) -- ($\trafficfactor*(0,-0.125) + (-0.2,0)$) --%
                  ($\trafficfactor*(0,-0.2) + (-0.1,0)$) -- cycle;% arrow backward
          }{}%
        }%
      }% end \ifdefstring{\face}
    }{%
      \ifdefstring{\type}{loop}{% type loop
        \path[line width=1pt] ($\trafficfactor*(0,-0.175)$) -- ++(-0.0625,-0.0625) -- ++(-0.2,0) -- ++(-0.125,0.125) -- ++(-0.1,0) -- ++(0,-0.125) -- ++(0.1,0) -- ++(0.125,0.125) -- ++(0.2,0) -- ++(0.125,-0.125) -- ++(0.2,0) -- ++(0.125,0.125) -- ++(0.1,0) -- ++(0,-0.125) -- ++(-0.1,0) -- ++(-0.125,0.125) -- ++(-0.2,0) -- cycle;% loop marker
      }{% error message
        \pgfkeys{/errors/unknown choice value={/tikz/trackschematic/trafficcontrol/transmitter/type}{“balise“ OR “loop“ as key required}}%
      }%
    }% end \ifdefstring{\type}
    %% label
    \ifdefstring{\labelcontent}{}{}{% label NOT empty
      \tikzset{every node/.style={font=\sffamily,text=\foreground}};%
      \coordinate (ts-tm-l) at ($\trafficfactor*(0,0.25)$);%
      \ifdefstring{\labelcoord}{(none)}{}{% initialize if NOT default
        \gettikzxy{\labelcoord}{\labelcoordX}{\labelcoordY}%
        \coordinate (ts-tm-l) at ($(ts-tm-l) + (\labelcoordX,\labelcoordY)$);%
      }%
      \node at (ts-tm-l) {\footnotesize \labelcontent};%
    }%
  }},% end pics/transmitter/.style args={#1/#2/#3}
}%
%
%%%%%%%%%%%%%%%
% symbol view point
%%%%%%%%%%%%%%%
% command
\newcommand\viewpoint{}% just for safety
\def\viewpoint[#1]#2(#3){% \viewpoint[options] at (coord);
  \pic[#1] at (#3) {view_point={#2}};% symbol
}%
% symbol definition
\tikzset{%
  pics/view_point/.default=,%
  pics/view_point/.style args={#1}{code={%
    %% face setup
    \ifdefstring{\face}{forward}{% face
      \pgfmathsetmacro{\facefactor}{1}%
    }{%
      \ifdefstring{\face}{backward}{% face
        \pgfmathsetmacro{\facefactor}{-1}%
      }{% error message
        \pgfkeys{/errors/unknown choice value={/tikz/trackschematic/face}{“forward“ OR “backward“ as key required}}%
      }%
    }% end \ifdefstring{\face}
    %% traffic practice setup
    \ifdefstring{\trafficpractice}{left}{% branch
      \pgfmathsetmacro{\trafficfactor}{-1}%
    }{%
      \ifdefstring{\trafficpractice}{right}{% branch
        \pgfmathsetmacro{\trafficfactor}{1}%
      }{% error message
        \pgfkeys{/errors/unknown choice value={/tikz/trackschematic/trafficcontrol/traffic practice}{“left“ OR “right“ as key required}}%
      }%
    }% end \ifdefstring{\trafficpractice}
    %% arrow
    \path[draw=\foreground,<-,>=latex,line width=1pt]%
      ($\facefactor*\trafficfactor*(0,-0.1)$) -- ++($\facefactor*\trafficfactor*(0,-0.3)$) -- ++($\facefactor*(0.2,0)$);% arrow
    %% eye
    \filldraw[\foreground] ($\facefactor*(0.4,0) + \facefactor*\trafficfactor*(0,-0.4)$) circle (0.1);% pupil
    \path[draw=\foreground, line width=1pt]% eye contour
      ($\facefactor*(0.4, 0) + \facefactor*\trafficfactor*(0,-0.15)$) .. controls%
      ($\facefactor*(0.25,0) + \facefactor*\trafficfactor*(0,-0.25)$) and%
      ($\facefactor*(0.25,0) + \facefactor*\trafficfactor*(0,-0.55)$) ..%
      ($\facefactor*(0.4, 0) + \facefactor*\trafficfactor*(0,-0.65)$) .. controls%
      ($\facefactor*(0.55,0) + \facefactor*\trafficfactor*(0,-0.55)$) and%
      ($\facefactor*(0.55,0) + \facefactor*\trafficfactor*(0,-0.25)$) ..%
      ($\facefactor*(0.4, 0) + \facefactor*\trafficfactor*(0,-0.15)$) --cycle;% eye contour
  }},%
}%
%
%%%%%%%%%%%%%%%
% symbol end of authority marker
%%%%%%%%%%%%%%%
% command
\newcommand\movementauthority{}% just for safety
\def\movementauthority[#1]#2(#3)#4(#5){% \movementauthority[options] at (coord) label (name);
  \pic[#1] at (#3) {movement_authority_marker={#2/#4/#5}}% symbol
}%
% symbol definition
\tikzset{%
  pics/movement_authority_marker/.default=,%
  pics/movement_authority_marker/.style args={#1/#2/#3}{code={%
    %% settings
    \def\coordcommand{#1}% beware of leading and tailing spaces!
    \def\labelcommand{#2}% beware of leading and tailing spaces!
    \def\labelcontent{#3}%
    %
    %% traffic practice setup
    \ifdefstring{\trafficpractice}{left}{% branch
      \pgfmathsetmacro{\trafficfactor}{-1}%
    }{%
      \ifdefstring{\trafficpractice}{right}{% branch
        \pgfmathsetmacro{\trafficfactor}{1}%
      }{% error message
        \pgfkeys{/errors/unknown choice value={/tikz/trackschematic/trafficcontrol/traffic practice}{“left“ OR “right“ as key required}}%
      }%
    }% end \ifdefstring{\trafficpractice}
    %% face setup
    \ifdefstring{\face}{forward}{% face
      \pgfmathsetmacro{\facefactor}{1}%
      \def\align{left}%
    }{%
      \ifdefstring{\face}{backward}{% face
        \pgfmathsetmacro{\facefactor}{-1}%
        \def\align{right}%
      }{% error message
        \pgfkeys{/errors/unknown choice value={/tikz/trackschematic/face}{“forward“ OR “backward“ as key required}}%
      }%
    }% end \ifdefstring{\face}
    \tikzset{every path/.style={draw=\foreground,line width=1pt}};%
    %% marker
      \path ($\trafficfactor*\facefactor*(0,-0.55) + \facefactor*(0.3,0)$) rectangle%
          ++($\trafficfactor*\facefactor*(0, 0.3) + \facefactor*(0.3,0)$);% 
    %% arrow
    \path[draw=\foreground,<-,>=latex,line width=1pt]%
      ($\facefactor*\trafficfactor*(0,-0.1)$) -- ++($\facefactor*\trafficfactor*(0,-0.3)$) -- ++($\facefactor*(0.2,0)$);% arrow
    %% label
    \ifdefstring{\labelcontent}{}{}{% label NOT empty
      \tikzset{every node/.style={font=\sffamily,text=\foreground}};%
      \coordinate (ts-ma-l) at ($\trafficfactor*\facefactor*(0,-0.6) + \facefactor*(0.3,0)$);%
      \ifdefstring{\labelcoord}{(none)}{}{% initialize if NOT default
        \gettikzxy{\labelcoord}{\labelcoordX}{\labelcoordY}%
        \coordinate (ts-ma-l) at ($(ts-ma-l) + (\labelcoordX,\labelcoordY)$);%
      }%
      \node[\align] at (ts-ma-l) {\footnotesize \labelcontent};%
    }%
  }},% end pics/movement_authority_marker/.style args={#1/#2/#3}
}%
%
%%%%%%%%%%%%%%%
% symbol braking point
%%%%%%%%%%%%%%%
% command
\newcommand\brakingpoint{}% just for safety
\def\brakingpoint[#1]#2(#3)#4(#5){% \brakingpoint[options] at (coord) label (name);
  \pic[#1] at (#3) {braking_point_marker={#2/#4/#5}}% symbol
}%
% symbol definition
\tikzset{%
  pics/braking_point_marker/.default=,%
  pics/braking_point_marker/.style args={#1/#2/#3}{code={%
    %% settings
    \def\coordcommand{#1}% beware of leading and tailing spaces!
    \def\labelcommand{#2}% beware of leading and tailing spaces!
    \def\labelcontent{#3}%
    %
    %% traffic practice setup
    \ifdefstring{\trafficpractice}{left}{% branch
      \pgfmathsetmacro{\trafficfactor}{-1}%
    }{%
      \ifdefstring{\trafficpractice}{right}{% branch
        \pgfmathsetmacro{\trafficfactor}{1}%
      }{% error message
        \pgfkeys{/errors/unknown choice value={/tikz/trackschematic/trafficcontrol/traffic practice}{“left“ OR “right“ as key required}}%
      }%
    }% end \ifdefstring{\trafficpractice}
    %% face setup
    \ifdefstring{\face}{forward}{% face
      \pgfmathsetmacro{\facefactor}{1}%
      \def\align{left}%
    }{%
      \ifdefstring{\face}{backward}{% face
        \pgfmathsetmacro{\facefactor}{-1}%
        \def\align{right}%
      }{% error message
        \pgfkeys{/errors/unknown choice value={/tikz/trackschematic/face}{“forward“ OR “backward“ as key required}}%
      }%
    }% end \ifdefstring{\face}
    \tikzset{every path/.style={draw=\foreground,line width=1pt}};%
    %% marker
      \path ($\trafficfactor*\facefactor*(0,-0.4) + \facefactor*(0.3,0)$) --%
          ++($\trafficfactor*\facefactor*(0,-0.15) + \facefactor*(0.25,0)$) --%
          ++($\trafficfactor*\facefactor*(0, 0.3)$) -- cycle;% signal marker
    %% arrow
    \path[draw=\foreground,<-,>=latex,line width=1pt]%
      ($\facefactor*\trafficfactor*(0,-0.1)$) -- ++($\facefactor*\trafficfactor*(0,-0.3)$) -- ++($\facefactor*(0.2,0)$);% arrow
    %% label
    \ifdefstring{\labelcontent}{}{}{% label NOT empty
      \tikzset{every node/.style={font=\sffamily,text=\foreground}};%
      \coordinate (ts-bp-l) at ($\trafficfactor*\facefactor*(0,-0.6) + \facefactor*(0.3,0)$);%
      \ifdefstring{\labelcoord}{(none)}{}{% initialize if NOT default
        \gettikzxy{\labelcoord}{\labelcoordX}{\labelcoordY}%
        \coordinate (ts-bp-l) at ($(ts-bp-l) + (\labelcoordX,\labelcoordY)$);%
      }%
      \node[\align] at (ts-bp-l) {\footnotesize \labelcontent};%
    }%
  }},% end pics/braking_point_marker/.style args={#1/#2/#3}
}%
%
%%%%%%%%%%%%%%%
% symbol danger point
%%%%%%%%%%%%%%%
% command
\newcommand\dangerpoint{}% just for safety
\def\dangerpoint[#1]#2(#3)#4(#5){% \dangerpoint[options] at (coord) label (name);
  \pic[#1] at (#3) {danger_point_marker={#2/#4/#5}}% symbol
}%
% symbol definition
\tikzset{%
  pics/danger_point_marker/.default=,%
  pics/danger_point_marker/.style args={#1/#2/#3}{code={%
    %% settings
    \def\coordcommand{#1}% beware of leading and tailing spaces!
    \def\labelcommand{#2}% beware of leading and tailing spaces!
    \def\labelcontent{#3}%
    %
    %% traffic practice setup
    \ifdefstring{\trafficpractice}{left}{% branch
      \pgfmathsetmacro{\trafficfactor}{-1}%
    }{%
      \ifdefstring{\trafficpractice}{right}{% branch
        \pgfmathsetmacro{\trafficfactor}{1}%
      }{% error message
        \pgfkeys{/errors/unknown choice value={/tikz/trackschematic/trafficcontrol/traffic practice}{“left“ OR “right“ as key required}}%
      }%
    }% end \ifdefstring{\trafficpractice}
    %% face setup
    \ifdefstring{\face}{forward}{% face
      \pgfmathsetmacro{\facefactor}{1}%
      \def\align{left}%
    }{%
      \ifdefstring{\face}{backward}{% face
        \pgfmathsetmacro{\facefactor}{-1}%
        \def\align{right}%
      }{% error message
        \pgfkeys{/errors/unknown choice value={/tikz/trackschematic/face}{“forward“ OR “backward“ as key required}}%
      }%
    }% end \ifdefstring{\face}
    \tikzset{every path/.style={draw=\foreground,line width=1pt}};%
    %% marker
      \path ($\trafficfactor*\facefactor*(0,-0.4)$) --%
          ++($\trafficfactor*\facefactor*(0,-0.1) + \facefactor*(-0.1,0)$) --%
          ++($\trafficfactor*\facefactor*(0,-0.1) + \facefactor*( 0.1,0)$) --%
          ++($\trafficfactor*\facefactor*(0, 0.1) + \facefactor*( 0.1,0)$) -- cycle;% sign
    %% arrow
    \path[draw=\foreground,<-,>=latex,line width=1pt]%
      ($\facefactor*\trafficfactor*(0,-0.1)$) -- ++($\facefactor*\trafficfactor*(0,-0.25)$);% arrow
    %% label
    \ifdefstring{\labelcontent}{}{}{% label NOT empty
      \tikzset{every node/.style={font=\sffamily,text=\foreground}};%
      \coordinate (ts-dp-l) at ($\trafficfactor*\facefactor*(0,0.25)$);%
      \ifdefstring{\labelcoord}{(none)}{}{% initialize if NOT default
        \gettikzxy{\labelcoord}{\labelcoordX}{\labelcoordY}%
        \coordinate (ts-dp-l) at ($(ts-dp-l)+(\labelcoordX,\labelcoordY)$);%
      }%
      \node at (ts-dp-l) {\footnotesize \labelcontent};%
    }%
  }},% end pics/danger_point_marker/.style args={#1/#2/#3}
}%
%
%%%%%%%%%%%%%%%
% symbol route
%%%%%%%%%%%%%%%
% command
\newcommand\route{}% just for safety
\def\route[#1]#2(#3){% \route[options] at (coord);
  \pic[#1] at (#3) {route={#2}}% symbol
}%
% symbol definition
\tikzset{%
  pics/route/.default=,%
  pics/route/.style args={#1}{code={%
    %% settings
    \def\coordcommand{#1} % beware of leading and tailing spaces!
    %% face setup
    \ifdefstring{\face}{forward}{% face
      \pgfmathsetmacro{\facefactor}{1}%
    }{%
      \ifdefstring{\face}{backward}{% face
        \pgfmathsetmacro{\facefactor}{-1}%
      }{% error message
        \pgfkeys{/errors/unknown choice value={/tikz/trackschematic/face}{“forward“ OR “backward“ as key required}}%
      }%
    }% end \ifdefstring{\face}
    %% symbol
    \fill[\foreground] ($\facefactor*(-0.175,0)+(0,-0.15)$) --%
                       ($\facefactor*(-0.175,0)+(0, 0.15)$) --%
                       ($\facefactor*( 0.175,0)+(0, 0   )$) -- cycle;%
  }},%
}%
%
%%%%%%%%%%%%%%%
\endinput%
%